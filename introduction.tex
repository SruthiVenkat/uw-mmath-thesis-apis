%======================================================================
\chapter{Introduction}
%======================================================================
In the beginning, there was $\pi$:

\begin{equation}
   e^{\pi i} + 1 = 0  \label{eqn_pi}
\end{equation}
A \gls{computer} could compute $\pi$ all day long. In fact, subsets of digits of $\pi$'s decimal approximation would make a good source for psuedo-random vectors, \gls{rvec} . 

%----------------------------------------------------------------------
\section{State of the Art}
%----------------------------------------------------------------------

See equation \ref{eqn_pi} on page \pageref{eqn_pi}.\footnote{A famous equation.}

\section{Some Meaningless Stuff}

The credo of the \gls{aaaaz} was, for several years, several paragraphs of gibberish, until the \gls{dingledorf} responsible for the \gls{aaaaz} Web site realized his mistake:

"Velit dolor illum facilisis zzril ipsum, augue odio, accumsan ea augue molestie lobortis zzril laoreet ex ad, adipiscing nulla. Veniam dolore, vel te in dolor te, feugait dolore ex vel erat duis nostrud diam commodo ad eu in consequat esse in ut wisi. Consectetuer dolore feugiat wisi eum dignissim tincidunt vel, nostrud, at vulputate eum euismod, diam minim eros consequat lorem aliquam et ad. Feugait illum sit suscipit ut, tation in dolore euismod et iusto nulla amet wisi odio quis nisl feugiat adipiscing luptatum minim nisl, quis, erat, dolore. Elit quis sit dolor veniam blandit ullamcorper ex, vero nonummy, duis exerci delenit ullamcorper at feugiat ullamcorper, ullamcorper elit vulputate iusto esse luptatum duis autem. Nulla nulla qui, te praesent et at nisl ut in consequat blandit vel augue ut.