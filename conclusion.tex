%======================================================================
\chapter{Conclusion}
%======================================================================
The goal of this work was to enable 1) library developers to make better
decisions when designing new APIs, pruning or modifying unused APIs and to refactor their
libraries; and 2) client developers to make better decisions about library
upgrades and breaking changes.

Further user evaluations of our VizAPI tool following
existing techniques~\cite{merino18:_system_liter_review_softw_visual_evaluat}; in
particular, experiments, where users perform software
understanding and maintenance tasks can establish and improve the
effectiveness of VizAPI.


\subsection{Threats to Validity}
Our threats to validity include the usual threat to external validity
of insufficient sample size or variety---many of our seed libraries
are JSON parser/generators, although our transitive closures result in
a wider range of domains.

There is also the construct validity issue that tests may
not adequately represent actual client behaviours. However, our use of both static
and dynamic information addresses this issue. Specifically, because we use
class hierarchy analysis for our static analysis, our visualization will present
all possible static calls---possibly too many. 
That is, the main hazard with static analysis is that our visualization may include more
static edges than are actually possible. Some of those edges could be ruled out by a more
precise call graph. Reflection aside, no static edges
are missing (our approach is ``soundy''~\cite{livshits15:_in_defen_sound} with respect to static information). On the other hand, dynamic edges have actually been observed
on some execution; better tests could yield more dynamic edges. But even if
a dynamic edge is missing, there will be a static edge if the behaviour is possible.

We may have missed other categories of bypass patterns---though we believe
that we have chosen at least a representative sample of mechanisms to ensure
modularity. 

Finally, our results apply best to Java-like languages, and
may vary dramatically for other languages.

\subsection{Actionable Outcomes}
Based on our exploration of API uses and mis-uses, we make some recommendations
for both API and language/analysis designers.

\begin{enumerate}
\item API scope: given that both we and Thummalapenta and Xie~\cite{thummalapenta08:_spotw} find
that APIs are sparsely used, API designers can seek to prune unused APIs;
\item Refactoring: our results show that library fission is useful, i.e., some existing APIs can be split into loosely-connected parts, reducing effective API surface and potentially developer cognitive burden;
\item Deprecation: one could investigate the scope for aggressive deprecation of unused APIs in released libraries, giving more liberty to API designers to modify their code;
\item Modularity: API and language designers can be confident that stated encapsulation boundaries are respected.
\end{enumerate}
%Encapsulation is a fundamental building block for software systems. API boundaries are an important way to enforce encapsulation, and our results show how developers interact with API boundaries in practice.

%% \item Automatically enforce encapsulation boundaries i.e. flag code that does things it shouldn't.
%% \begin{enumerate}
%% \item Upstream can provide wider published APIs if needed
%% \item Downstream can refactor to use narrower APIs if required
%% \item Can provide shims for downstream to use until upstream widens.
%% \end{enumerate}
%% \item Restrict the declared API to an actually-used subset and support deprecation of unused API bits.
%% \end{enumerate}


