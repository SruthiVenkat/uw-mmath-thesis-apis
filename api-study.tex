%======================================================================
\chapter{API Usage Study}
\label{sec:apiusage}
%======================================================================
We aim to take a snapshot of where current software practices stand: we investigate how APIs are used and misused. Our goal is to investigate API usage patterns, both conceptually and in practice.
Such patterns can provide interesting hints for API designers in the future.

We begin by defining two terms that we use in this work, followed by an example. 
A library's \emph{intended API surface} is the set of
classes, methods and fields that it expects clients to use. Other
classes, methods and fields are \emph{internal} to the library. Clients using internal parts of the library perform
some sort of \emph{bypass}. On
the other hand, a client uses an \emph{actual API surface} of the
library. In the absence of bypasses, the actual surface is a subset of
the declared surface.


\begin{figure}[h]
 \begin{center}
  \begin{tikzpicture}
    \node (client-a) at (3.3, 0.5) {$u_a$};
    \node (client-b) at (3, 1.7) {$u_b$};

    \node (class-1) at (-1.5, -1.3) {\texttt{C}$_1$};
    \node (class-2) at (1.7, -1.2) {\texttt{C}$_2$};
    \node (class-3) at (1.5, 1) {\texttt{C}$_3$};
    \node (class-4) at (-1.5, 1) {\texttt{C}$_4$};
    \node[draw] (class-5) at (0,0) {\texttt{C}$_5$ (internal)};

    \draw (-2.1, -1.6) rectangle (2.1, 1.5);

    \node (lib) at (-1.4, 1.7) {Library $L$};

    \draw[-Latex] (client-a) -- (class-2);
    \draw[-Latex] (client-a) -- (class-3);

    \draw[-Latex] (client-b) -- (class-3);
    \draw[-Latex] (client-b) -- (class-4);
  \end{tikzpicture}
  \caption{API surface of library $L$, which exports classes \texttt{C}$_1$ through \texttt{C}$_4$. Class \texttt{C}$_5$ is internal and direct calls to it are not allowed. Client $u_a$ uses classes \texttt{C}$_3$ and \texttt{C}$_2$, while client $u_b$ uses classes \texttt{C}$_4$ and \texttt{C}$_3$.}
  \label{fig:api-surface}
 \end{center}
\end{figure}


Figure~\ref{fig:api-surface} illustrates the situation where library
$L$ is used by clients (users) $u_a$ and $u_b$. $L$'s intended API surface
includes classes \texttt{C}$_1$ through \texttt{C}$_4$, while class
\texttt{C}$_5$ is internal to $L$, and uses of it involve bypasses. From Figure~\ref{fig:api-surface},
we can see that the actual API surface includes classes
\texttt{C}$_2$ through \texttt{C}$_4$, and we don't know about whether
\texttt{C}$_1$ is used by any extant client.


\section{Methodology}
\label{sec:methodology}

We now describe the implementation of our tool and our benchmark selection process.

\subsection{Static Analysis and Instrumentation}
\label{sec:methodology}

We use Javassist~\cite{chiba00:_load_struc_reflec_java} and perform class hierarchy analysis on clients
and create a static call graph. We collect data about client usages of libraries by running client
test suites under instrumentation. The instrumentation records API
uses which cross client/library boundaries, closely mirroring the API
usage patterns that we describe in
Section~\ref{sec:patterns}. We also use Javassist for instrumentation,
and the build system of each project (Maven) is used to run its
tests and we obtain dynamic call graphs. 
\\

\begin{figure*}[h]
 \begin{center}
\resizebox{0.9\textwidth}{!}{
  \begin{tikzpicture}
    \node[block] (client) {client};
    \node[block,below=1cm of client] (library) {library};

    \draw (library) -- node[left] (depends) {depends on} (client);

    \node[above left=.75em of client] (ja) {\begin{minipage}{7em} modify \\with Javassist \end{minipage}};
    \draw[-Latex] (ja) -> (client);
    \draw[-Latex] (ja) to [->,bend right=35] (library.west);

    \node[block, above right=2em of client,xshift=-2em] (olib) {other library};
    \draw (client) -- node[right,xshift=.1em] (also) {also depends on} (olib);

    \node[oval,right=of depends] (test) {maven/gradle: run tests};

    \draw[-Latex] (client) to [->,bend left=15] (test);

    \node[block, right=10em of client] (output) {test output};
    \node[block, right=10em of library] (raw) {raw API usage info};

    \draw[-Latex] (test) to (output);
    \draw[-Latex] (test) to (raw);

    \node[oval, right=of raw] (Py) {Python scripts};
    \draw[-Latex] (raw) to (Py);

    \node[block, right=1em of Py] (viz) {D3 visualizations};
    \draw[-Latex] (Py) to (viz);
  \end{tikzpicture}
}
  \caption{Our instrumentation workflow. Using Javassist, we analyze and instrument clients and run their test suites. (We process the generated data with Python scripts to create D3 visualizations for VizAPI.)}
  \label{fig:workflow}
 \end{center}
\end{figure*}

Figure~\ref{fig:workflow} summarizes our instrumentation and
data capture workflow. We next describe our instrumentation implementation in detail.

We identify interactions across the client/library boundaries by inspecting JAR files of
each software component to obtain a list of classes for every component. We associate classes 
and their members to components based on these lists. Since the JAR files contain source code,
we ensure that none of the library uses meant solely for unit testing are captured.

\paragraph{Vanilla invocations}
The standard case is simple. At every invoke instruction in every
loaded method which transfers control between the client and the
library, we add code to record that invoke by incrementing a counter.
We handle both static and virtual (including special, virtual,
interface, and dynamic) calls. Crossing the client/library boundary
includes conventional calls from the client to the library as well as callbacks from the library to the client.  

\paragraph{Field accesses}
We capture direct (field sets and gets) and reflective (via invocations of
\texttt{java.lang.reflect.Field.get()} and \texttt{.set()}) field
accesses.

\paragraph{Dynamic proxies and reflective calls}
We specially handle invocations of the distinguished method 
\texttt{java.lang.reflect.Method.invoke()} method used to invoke dynamic proxies and reflective calls, recording
details of the calls that we intercept. 
We identify dynamic proxies by checking whether the invocation 
of \texttt{Method.invoke()} originates from a class that implements 
\texttt{java.lang.reflect.InvocationHandler}. If so,
we inspect the call stack to find the caller and callee of 
\texttt{Method.invoke()} and record the call if it crosses the client/library boundary. 
All other calls to \texttt{Method.invoke()} are standard reflective calls, 
and we record the respective callers and callees.
(We also specifically ignore calls to \texttt{Method.invoke()} made by the Maven surefire plugin
as it runs tests.)
% \todo[inline]{When running tests, the Maven surefire plugin uses the invoke method too,
% which is ignored. Should we talk about that?}

Instrumenting methods also allows us to capture several other library uses,
as we describe below.

\paragraph{Class usages}
We capture reflective uses of the \texttt{Class} object by intercepting calls to
\texttt{java.lang.Class.forName()} and \texttt{java.lang.ClassLoader.loadClass()}.

\paragraph{Service Loaders} We are particularly interested in bypasses of 
services using \texttt{ServiceLoader}. Before the instrumentation, we record a list 
of services and their implementations by inspecting files in \texttt{src/main/resources/META-INF/services}.
With this information, we look for service bypasses which are direct uses of service implementation 
classes in clients either through instantiations, casts or reflection. We also intercept calls 
to method \texttt{load()} in classes with name \texttt{Service*Loader} and record any calls to methods beyond 
the published interface.

\paragraph{setAccessible()} 
Java provides the \texttt{setAccessible()} method to allow reflective access to class members despite
access modifiers. After a call to this method, the program may then (subject to security manager restrictions)
reflectively access the class member.
We thus record calls to \texttt{setAccessible()} along with the previous visibility of the class member.

\paragraph{Annotations} 
We have a quasi-static approach for finding class, field and method
annotations: we observe all annotations for a class or class member
when it is loaded, and record cases where a class or member declares an
annotation from the library of interest. We also record an association
between the class and its memers' annotations.

\paragraph{Inheritance and interface implementation} At load time,
we also record information about all superclasses and implemented interfaces
that cross the library/client barrier.

\paragraph{Instantiations and casts} We also instrument the
\texttt{NewExpr} and \texttt{Cast} bytecodes to record library/client 
instantiations and casts.



\subsection{Benchmark Selection}
\label{sec:benchmark}
Our benchmark set consists of 11 libraries and 90 clients. For libraries, we pick the most popular Maven 
repositories in different categories such as logging, json parsing and databases. Table~\ref{tab:libs} presents our set of libraries. We measured lines of code (kLOC) using SLOCcount\footnote{\url{https://dwheeler.com/sloccount/}} and number of classes by building libraries and counting resulting \texttt{.class}es. A project uses ServiceLoaders if it has a \texttt{META-INF/services} directory and Java 9 modules if it has a \texttt{module-info.java} file. A library is an OSGi component if it contains a \texttt{MANIFEST.MF} file in its build output\footnote{Some libraries (for example, \emph{connector-j}) create OSGi metadata during the build, so we look for the metadata in the output, not in the source.}, and this manifest contains \texttt{Export-Package} declarations. 

\begin{table}[h]
\begin{center}
\caption{\label{tab:libs}Libraries that we investigated for API usage and mis-usage patterns}
\begingroup\scriptsize	
\hskip-2.0cm
\begin{tabular}{l!{\color{verylightgray}\vrule}cl!{\color{verylightgray}\vrule}rr!{\color{verylightgray}\vrule}ccc}
& & & non-test &  & Service & Java 9 &  \\
Library & version & description   & kLOC     & \# classes  &  Loader  & modules & OSGi \\ \arrayrulecolor{verylightgray}\hline
commons-collections4 & 4.4 & data structure implementations & 28.9 & 524 &&&\checkmark\\
commons-io & 2.8.0 & IO functionality library & 12.6 & 182& & &\checkmark\\
joda-time & 2.10.10 & date and time handling library & 28.9 & 247&&& \checkmark \\
slf4j-api & 1.7.9 & logging library & 1.5 & 28 & & & \checkmark\\
jsoup & 1.13.1 & HTML parser & 12.5 & 249&&&\checkmark\\
fastjson & 1.2.76 & json parser/generator & 43.6 & 260 &\checkmark&\\ 
gson & 2.8.8 & json parser/generator & 14.4&  182 && \checkmark&\checkmark\\
json & 20210307 & json parser/generator & 11.8 & 27& & \\
jackson-core & 2.12.3 & json parser/generator & 27.1 & 124 & \checkmark&  \checkmark&\\
jackson-databind & 2.12.3 & bindings for json parser/generator & 68.2 & 700 & \checkmark & \checkmark&\\ 
h2 & 1.4.200 & database & 147.2 & 1010 & \checkmark & & \checkmark \\
\end{tabular}
\endgroup
\end{center}
\end{table}

We use the libraries.io dataset\footnote{\url{https://libraries.io/}} to construct a dependency graph and look for the most used upstream components (highest number of other components depends on [any version of] those), 
and the top downstream components (clients). We exclude any clients that have less than 10 stars or less than 10 forks on Github.  Apart from this, we also pick a subset of projects from the
Duets benchmarks~\cite{durieux21}. We exclude components that do not contain unit tests, components that use our chosen library only for testing
and components that declare the library as a dependency in their POM file, but do not actually use it. Our benchmark set contains both Maven single module and multi-module components.

We executed each of the clients' test suites to collect data about how the clients use all of their dependencies; our data therefore includes not just interactions between our clients and the 11 libraries sampled, but also ``bycatch''---that is, other libraries that are also called by the clients (``also depends on'' in Figure~\ref{fig:workflow}) and the libraries. The total static transitive closure of dependencies of our clients includes 4297 components.

Collecting execution data from programs is more challenging than it seems: downloading software and collecting static numbers is fairly straightforward, but running this software to instrument it involves fixing numerous uninteresting environment glitches which nevertheless block progress---even in the stable environment of a continuous integration system at a large software company, Kerzazi et al~\cite{kerzazi14:_why_do_autom_build_break} found that 17.9\% of builds break, and our context is even more challenging.

We have made our data publicly available\footnote{\url{https://zenodo.org/record/6951140}}.

\section{Results}
\label{sec:results}

We look for our usage patterns in our benchmark collection of 11 libraries (Table~\ref{tab:libs}).

\subsection{API bypasses}
We first investigate the use of API bypasses in our clients. This is useful to library developers if they want to identify which parts of their libraries that are supposed to be internal are actually used by clients. This can help them with API design in future versions. 

In Section~\ref{sec:classification}, we enumerate three broad categories
of bypasses: access modifiers, modularity conventions, and service loaders. We discuss each of them in turn.


\paragraph{Access modifiers and reflection}
% latex table generated in R 4.2.0 by xtable 1.8-4 package
% Tue Aug 23 13:16:06 2022
\begin{table}[ht]
\centering
\begingroup\small
\begin{tabular}{llllr}
  \hline
Library & Client & Member & Visibility & Count \\ 
  \hline
libthrift & benchmark-thrift & Field & private & 1 \\ 
  spring-context & fastjson & Constructor & private & 1 \\ 
  javax.servlet-api & fastjson & Field & private & 9 \\ 
  spring-context & fastjson & Field & private & 22 \\ 
  javax.servlet-api & fastjson & Method & public & 17 \\ 
  spring-beans & fastjson & Method & public & 11 \\ 
  spring-context & fastjson & Method & public & 38 \\ 
  spring-web & fastjson & Method & public & 3 \\ 
  jsoup & JsoupXpath & Method & protected & 1 \\ 
  rocketmq-common & rocketmq-acl & Field & private & 7 \\ 
   \hline
\end{tabular}
\endgroup
\caption{\label{tab:set-accessible-client-to-lib}setAccessible Calls---Client to Library} 
\end{table}

% latex table generated in R 4.1.1 by xtable 1.8-4 package
% Fri Oct 22 21:21:27 2021
\begin{table*}[ht]
\centering
\caption{\label{tab:set-accessible-lib-to-client}setAccessible Calls---Library to Client} 
\begingroup\scriptsize
\begin{tabular}{ll!{\color{verylightgray}\vrule}rrr!{\color{verylightgray}\vrule}rrrr!{\color{verylightgray}\vrule}r}
Library & Client & Fields & Constructors & Methods & private & default & protected & public & total \\ 
 \arrayrulecolor{verylightgray}\hline
  fastjson              & rocketmq-*                & 159 & 37 & 304 & 158 &1 &  1  & 340 & 500 \\ 
  jackson-databind      & capitalone.dashboard:core & 35 & 12 & 0    & 34  &1 &  0  & 12 & 47 \\ 
  jackson-databind      & ez-vcard                  & 0 & 2 & 0      & 0   &0 &  0  & 2 & 2 \\ 
  jackson-databind      & swagger-codegen           & 117 & 1 & 34   & 0   &0 &  0  & 152 & 152 \\ 
  gson                  & capitalone.dashboard:core & 220 & 24 & 0   & 196 &9 &  17 & 22 & 244 \\ 
  groovy                & logback-classic           & 0 & 5 & 0      & 0   &1 &  0  & 4 & 5 \\ 
  hk2-locator           & fastjson                  & 0 & 3 & 0      & 0   &0 &  0  & 3 & 3 \\ 
  mockito-all           & capitalone.dashboard:core & 2 & 6 & 0      & 2   &0 &  0  & 6 & 8 \\ 
  jmustache             & swagger-codegen           & 70 & 0 & 1     & 0   &0 &  0  & 71 & 71 \\ 
  jaxb-impl             & capitalone.dashboard:core & 28 & 0 & 0     & 0   &0 &  28 & 0 & 28 \\ 
  mirror                & vraptor                   & 1 & 0 & 25     & 1   &0 &  0  & 25 & 26 \\ 
  commons-lang3         & rocketmq-acl              & 1 & 0 & 0      & 1   &0 &  0  & 0 & 1 \\ 
  commons-lang3         & rocketmq-client           & 7 & 0 & 0      & 7   &0 &  0  & 0 & 7 \\ 
  cxf-rt-frontend-jaxrs & fastjson                  & 1 & 0 & 0      & 0   &0 &  1  & 0 & 1 \\ 
  rocketmq-common       & rocketmq-client           & 28 & 0 & 0     & 27  &0 &  1  & 0 & 28 \\ 
  spring-core           & capitalone.dashboard:core & 628 & 0 & 0    & 625 &3 &  0  & 0 & 628 \\ 
\end{tabular}
\endgroup
\end{table*}

Table~\ref{tab:set-accessible-client-to-lib}
shows selected uses of the reflection API \texttt{setAccessible()} to enable reflective access from clients to libraries,
and Table~\ref{tab:set-accessible-lib-to-client} shows selected uses of \texttt{setAccessible()} to enable reflective callbacks from libraries back to clients.
We can see that accesses to fields, constructors, and methods are all reasonably common, though some codebases only reflectively 
access fields, while others only access methods.

In our data, 1.9\% of the \texttt{setAccessible()} calls are 
used to ensure accessibility of a class member
in the same \texttt{rocketmq} project version 4.9.1 but in a different module (Maven module). 
For instance, \texttt{rocketmq-acl} makes private field \texttt{SendMessageRequestHeaderV2::a} from 
\texttt{rocketmq-common} accessible before accessing it. Such inter-module accesses are more controlled than client/library accesses since they are within the same project,
but are still a form of technical debt.

We investigated all \texttt{setAccessible()} calls on fields that belong to in \emph{rocketmq}. (\emph{rocketmq} is a client that calls \emph{fastjson}). These calls are made from the library to the client, i.e., accesses are from library \emph{fastjson} to client \emph{rocketmq} using callbacks. We see hundreds of \texttt{setAccessible()} calls being executed when the client tests are run. The \emph{rocketmq} source code shows 6 classes with calls to \texttt{setAccessible()}: a serialization/deserialization pair for the \texttt{CommandCustomHeader} class, 2 methods which appear to print out the state of an object for debugging or logging purposes, and 2 methods which store a reference to a specified field or which call a specified method, provided in those methods' parameters. We would not characterize the first 4 API bypasses as mis-uses, and it is not obvious that the last 2 are mis-uses either. Another interesting use of \texttt{setAccessible()} on a field is by \emph{benchmark-thrift} to access the field \texttt{maxLength\_}  which is a class member of the class \texttt{TFramedTransport} belonging to Apache \emph{thrift}. This is a private field and the default maximum length is overridden by the client.

We found that 23\% of the calls to \texttt{setAccessible()} were on methods. Of those, only 2 distinct methods that were reflectively invoked were previously not accessible. We manually inspected all of the reflective callers of these 2 methods. Protected method \texttt{Node.setParentNode()} in \emph{jsoup} is called from code in \texttt{org.seimicrawler.xpath} in \emph{JsoupXpath}. This appears to be test code committed to the main repository. The next method is the default-visibility method \texttt{getInstance()}. This method is used to obtain a singleton instance of {\tt ReflectionNavigator} from within the same project. The motivation for developers calling \texttt{setAccessible} within the same project is unclear to us, rather than modifying the code themselves. This work can help developers identify such instances and possibly refactor their code.

We do find that some methods are already accessible before being invoked, and some methods are made accessible but never actually invoked

An API bypass requires a \texttt{setAccessible()} call followed by an
actual reflective access.  As expected, we see a good overlap between the
\texttt{setAccessible()} calls to methods and reflective invocations.
We find that our benchmarks contain reflective invocations of 20 constructors following a call to \texttt{setAccessible}, which were not already public. Of the 20 callsites, some are generated by the
groovy dynamic language; some are for serialization and some instantiate objects where the constructor had default visibility. All of these reflective constructor invocations are callbacks from the library to the client: the library is providing a factory method to create instances of a client type that it has been supplied with. This appears to be an acceptable API use.
Tables~\ref{tab:refl-fields} and \ref{tab:refl-callbacks} show how many reflective usages are actually made.

Table~\ref{tab:refl-fields} shows some of our data on reflective field accesses. We observe that many of these reflective field accesses are for serialization. We also observe that in the case of fields, there are only a few calls that access fields that were previously not accessible.
% latex table generated in R 4.2.0 by xtable 1.8-4 package
% Wed Aug 24 16:58:20 2022
\begin{table}[ht]
\centering
\begingroup\small
\begin{tabular}{ll!{\color{verylightgray}\vrule}rrrr!{\color{verylightgray}\vrule}r}
Library & Client & default & private & protected & public & total \\ 
  \arrayrulecolor{verylightgray}\hline
vraptor & mirror & 0 & 1 & 0 & 0 & 1 \\ 
  dble & gson & 0 & 0 & 12 & 0 & 12 \\ 
  capitalone.dashboard:core & jackson-databind & 0 & 26 & 0 & 0 & 26 \\ 
  capitalone.dashboard:core & gson & 3 & 142 & 0 & 0 & 145 \\ 
  capitalone.dashboard:core & jaxb-impl & 0 & 0 & 4 & 0 & 4 \\ 
  capitalone.dashboard:core & mockito-all & 0 & 4 & 0 & 0 & 4 \\ 
  capitalone.dashboard:core & spring-beans & 1 & 21 & 0 & 0 & 22 \\ 
  capitalone.dashboard:core & spring-core & 2 & 157 & 0 & 0 & 159 \\ 
  benchmark-thrift & jcommander & 0 & 5 & 0 & 0 & 5 \\ 
  pagehelper-\ldots-autoconfigure & spring-beans & 0 & 2 & 0 & 0 & 2 \\ 
  MultiChainJavaAPI & gson & 21 & 0 & 0 & 0 & 21 \\ 
  motan-core & hessian & 0 & 0 & 2 & 0 & 2 \\ 
  crushpaper & hibernate-core & 0 & 33 & 0 & 0 & 33 \\ 
  dubbo-cluster & dubbo-common & 0 & 31 & 0 & 0 & 31 \\ 
  dubbo-common & gson & 0 & 20 & 0 & 0 & 20 \\ 
  dubbo-config-api & dubbo-common & 0 & 1 & 0 & 0 & 1 \\ 
  dubbo-metadata-api & gson & 0 & 22 & 0 & 0 & 22 \\ 
  dubbo-registry-api & gson & 2 & 0 & 0 & 0 & 2 \\ 
  rocketmq-broker & commons-lang3 & 0 & 8 & 0 & 0 & 8 \\ 
  rocketmq-client & commons-lang3 & 0 & 7 & 0 & 0 & 7 \\ 
  rocketmq-client & rocketmq-common & 0 & 27 & 1 & 0 & 28 \\ 
  rocketmq-common & rocketmq-acl & 0 & 7 & 0 & 0 & 7 \\ 
  rocketmq-common & rocketmq-remoting & 0 & 60 & 0 & 0 & 60 \\ 
  rocketmq-remoting & rocketmq-common & 0 & 23 & 0 & 0 & 23 \\ 
  rocketmq-store & rocketmq-common & 0 & 69 & 0 & 0 & 69 \\ 
  libthrift & benchmark-thrift & 0 & 1 & 0 & 0 & 1 \\ 
  mybatis & pagehelper & 0 & 1 & 0 & 0 & 1 \\ 
  thymeleaf & ognl & 0 & 0 & 0 & 1 & 1 \\ 
  thymeleaf & spring-expression & 0 & 0 & 0 & 1 & 1 \\ 
\end{tabular}
\endgroup
\caption{\label{tab:refl-fields}Reflection on fields} 
\end{table}


Table~\ref{tab:refl-callbacks} shows the reflective callback API usage pattern. We see that most accesses in this pattern are made on public methods. We examined the list of methods that are accessed using this usage pattern and this pattern is popular for logging, serialization and deserialization. We also observe a lot of getters and setters accessed this way.



%% latex table generated in R 4.2.0 by xtable 1.8-4 package
% Wed Aug 24 19:34:57 2022
\begin{table}[ht]
\centering
\caption{\label{tab:refl-inv}Reflective calls from client to library} 
\begingroup\small
\begin{tabular}{lllr}
  \hline
Library & Client & Visibility & Count \\ 
  \hline
libthrift & benchmark-thrift & private & 1 \\ 
  rocketmq-broker & commons-lang3 & private & 8 \\ 
  rocketmq-client & commons-lang3 & private & 7 \\ 
  dubbo-cluster & dubbo-common & private & 31 \\ 
  dubbo-config-api & dubbo-common & private & 1 \\ 
  capitalone.dashboard:core & gson & default & 3 \\ 
  MultiChainJavaAPI & gson & default & 21 \\ 
  dubbo-registry-api & gson & default & 2 \\ 
  capitalone.dashboard:core & gson & private & 142 \\ 
  dubbo-common & gson & private & 20 \\ 
  dubbo-metadata-api & gson & private & 22 \\ 
  dble & gson & protected & 12 \\ 
  motan-core & hessian & protected & 2 \\ 
  crushpaper & hibernate-core & private & 33 \\ 
  capitalone.dashboard:core & jackson-databind & private & 26 \\ 
  capitalone.dashboard:core & jaxb-impl & protected & 4 \\ 
  benchmark-thrift & jcommander & private & 5 \\ 
  vraptor & mirror & private & 1 \\ 
  mybatis & pagehelper & private & 1 \\ 
  rocketmq-common & rocketmq-acl & private & 7 \\ 
  rocketmq-client & rocketmq-common & private & 27 \\ 
  rocketmq-remoting & rocketmq-common & private & 23 \\ 
  rocketmq-store & rocketmq-common & private & 69 \\ 
  rocketmq-client & rocketmq-common & protected & 1 \\ 
  rocketmq-common & rocketmq-remoting & private & 60 \\ 
  capitalone.dashboard:core & spring-beans & default & 1 \\ 
  capitalone.dashboard:core & spring-beans & private & 21 \\ 
  pagehelper-spring-boot-autoconfigure & spring-beans & private & 2 \\ 
  capitalone.dashboard:core & spring-core & default & 2 \\ 
  capitalone.dashboard:core & spring-core & private & 157 \\ 
  jtds & druid & public & 1 \\ 
  java.security & fastjson & public & 1 \\ 
  javax.servlet-api & fastjson & public & 28 \\ 
  joda-time & fastjson & public & 2 \\ 
  slf4j-api & fastjson & public & 1 \\ 
  spring-* & fastjson & public & 84 \\ 
  flink-shaded-asm-7 & flink-core & public & 1 \\ 
  jsoup & JsoupXpath & public & 1 \\ 
   \hline
\end{tabular}
\endgroup
\end{table}



%% facts about methods:
%% # 827
%% protected 1 org.seimicrawler.xpath.core.node.Text$1::head(Lorg/jsoup/nodes/Node;I)V to org.jsoup.nodes.Node::setParentNode(Lorg/jsoup/nodes/Node;)V, looks like test code
%% default 3 com.sun.xml.bind:jaxb-core:2.2.11 to com.sun.xml.bind.v2.model.nav.ReflectionNavigator::getInstance()Lcom/sun/xml/bind/v2/model/nav/ReflectionNavigator;

%% many of the other calls are during serialization


%% Similarly, \texttt{setAccessible()} on methods in \emph{fastjson} at the behest of \emph{rocketmq} is on constructors, factory methods, and build methods, in the context of Java Beans. As with the fields, these calls essentially enable serialization and deserialization, and are made from class JavaBeanDeserializer.

%% \todo[inline]{Our hypothesis: setAccessible() on methods is used almost exclusively in tests and in serialization. We can verify this by filtering the setAccessible() method calls and excluding everything with ``Test'' and ``Bean'' in it.}
% latex table generated in R 4.2.0 by xtable 1.8-4 package
% Wed Aug 24 19:50:29 2022
\begin{table}[ht]
\centering
\begin{tabular}{ll!{\color{verylightgray}\vrule}rrrr!{\color{verylightgray}\vrule}r}
Library & Client & default & private & protected & public & total \\ 
  \arrayrulecolor{verylightgray}\hline
mirror & vraptor & 0 & 0 & 0 & 1 & 1 \\ 
  jboss-el-api\_3.0\_spec & vraptor & 0 & 0 & 0 & 4 & 4 \\ 
  dble & dble & 0 & 0 & 0 & 11 & 11 \\ 
  fastjson & dble & 0 & 0 & 0 & 12 & 12 \\ 
  junit & dble & 0 & 2 & 0 & 0 & 2 \\ 
  alipay-sdk-java & alipay-sdk-java & 0 & 0 & 0 & 63 & 63 \\ 
  jackson-databind & bandwidth-java-sdk & 0 & 0 & 0 & 2 & 2 \\ 
  fongo & capitalone.dashboard:core & 0 & 0 & 0 & 3 & 3 \\ 
  spring-beans & capitalone.dashboard:core & 0 & 0 & 0 & 7 & 7 \\ 
  spring-core & capitalone.dashboard:core & 0 & 0 & 0 & 2 & 2 \\ 
  spring-data-commons & capitalone.dashboard:core & 0 & 0 & 0 & 1 & 1 \\ 
  mirror & vraptor & 3 & 2 & 2 & 13 & 20 \\ 
  jboss-el-api\_3.0\_spec & vraptor & 0 & 0 & 0 & 4 & 4 \\ 
  logback-core & logback-classic & 0 & 0 & 0 & 13 & 13 \\ 
  janino & logback-classic & 0 & 0 & 0 & 3 & 3 \\ 
  slf4j-api & logback-classic & 0 & 0 & 0 & 1 & 1 \\ 
  spring-beans & druid & 0 & 0 & 0 & 4 & 4 \\ 
  cxf-core & fastjson & 0 & 0 & 0 & 1 & 1 \\ 
  spring-web & fastjson & 0 & 0 & 0 & 1 & 1 \\ 
  spring-beans & core & 0 & 0 & 0 & 7 & 7 \\ 
  spring-data-commons & core & 0 & 0 & 0 & 1 & 1 \\ 
  freemarker & ez-vcard & 0 & 0 & 0 & 64 & 64 \\ 
  querydsl-core & querydsl-sql & 0 & 0 & 1 & 2 & 3 \\ 
  groovy & json-path & 0 & 10 & 0 & 5 & 15 \\ 
  groovy & rest-assured & 0 & 34 & 1 & 141 & 176 \\ 
  springside-utils & springside-core & 0 & 0 & 0 & 1 & 1 \\ 
  junit & springside-metrics & 0 & 0 & 0 & 7 & 7 \\ 
  jackson-databind & swagger-codegen & 0 & 0 & 0 & 24 & 24 \\ 
  jmustache & swagger-codegen & 0 & 0 & 0 & 1 & 1 \\ 
  fastjson & rocketmq-* & 0 & 0 & 0 & 232 & 232 \\ 
  netty-all & rocketmq-remoting & 0 & 0 & 0 & 4 & 4 \\ 
\end{tabular}
\caption{\label{tab:refl-callbacks}Reflective Callbacks} 
\end{table}


\paragraph{Containers, modules, and modularity conventions}
For OSGi, none of our libraries are used by our clients in the context of OSGi containers, so the clients are free to violate OSGi access control. Our results show that, even though they can, they almost universally do not. The sole exception is a pair of calls from client \texttt{xsoup} to internal class \texttt{StringUtil} of library \texttt{jsoup}; the calling class in \texttt{xsoup} was copied from \texttt{jsoup} and still uses its internal helper functions. These calls violate both modularity conventions and OSGi export declarations. When we found these calls, we submitted a pull request\footnote{https://github.com/code4craft/xsoup/pull/53} duplicating the \texttt{jsoup} methods into \texttt{xsoup}, and it was quickly merged, showing that the \texttt{xsoup} developer was not in favour of violating modularity conventions.
% https://github.com/code4craft/xsoup/pull/53

Similarly, although we have Java 8 clients which can violate the unenforced module export rules of Java 9 libraries (they run in environments that don't enforce the rules), none of the clients do so. We believe that the most likely explanation is that such modularity violations are uncommon; however, it is also possible that Java 9 module definitions are too permissive and do not prevent calls that should be prevented.

Table~\ref{tab:internal} shows uses of packages labelled ``internal'' from outside
a given module. In some cases (for example, netty-buffer and netty-common),
the uses are across different modules in the same project. We looked
at one case, \emph{rocketmq-remoting} and \emph{netty}. In this case,
\emph{rocketmq-remoting} uses internal logging infrastructure from
\emph{netty} in its NettyLogger module; such a module might be
expected to be more closely coupled to its callee than other parts of
the code. On the other hand, the usage of \emph{groovy} internals in
\emph{rest-assured} would appear to be due to the choice of Groovy as
an implementation language, and thus compiler-generated references to
internals in the \emph{rest-assured} code.

\begin{table}[ht]
\centering
\caption{\label{tab:internal}Uses of .internal APIs externally}
\begingroup\scriptsize
\begin{tabular}{ll!{\color{verylightgray}\vrule}r}
To & From & count \\ 
 \arrayrulecolor{verylightgray}\hline
fastjson	&jersey-common	&1\\
netty-all	&rocketmq-remoting	&2\\
netty-common	&lettuce-core	&2\\
netty-common	&netty-buffer	&4\\
netty-common	&netty-transport	&1\\
json-path	&groovy	&13\\
rest-assured	&groovy	&147\\
rest-assured-common	&json-path	&2\\
rest-assured-common	&rest-assured	&2\\
rest-assured-common	&groovy	&15\\
jasperreports	&ecj	&11\\
rocketmq-remoting	&netty-all	&9\\
ecj	&jasperreports	&6\\
jersey-common	&fastjson	&1\\
jersey-common	&hk2-utils	&2\\
kotlin-stdlib	&okio	&3\\
jsoup	& xsoup	&2\\
mockito-core	&ognl	&1\\
mockito-core	&spring-expression	&1\\
mongo-java-driver	&fongo	&3\\
\end{tabular}
\endgroup
\end{table}

%% com.alibaba:fastjson	&org.glassfish.jersey.core:jersey-common	1\\
%% io.netty:netty-all	&org.apache.rocketmq:rocketmq-remoting	2\\
%% io.netty:netty-common	&io.lettuce:lettuce-core	2\\
%% io.netty:netty-common	&io.netty:netty-buffer	4\\
%% io.netty:netty-common	&io.netty:netty-transport	1\\
%% io.rest-assured:json-path	&org.codehaus.groovy:groovy	13\\
%% io.rest-assured:rest-assured	&org.codehaus.groovy:groovy	147\\
%% io.rest-assured:rest-assured-common	&io.rest-assured:json-path	2\\
%% io.rest-assured:rest-assured-common	&io.rest-assured:rest-assured	2\\
%% io.rest-assured:rest-assured-common	&org.codehaus.groovy:groovy	15\\
%% net.sf.jasperreports:jasperreports	&org.eclipse.jdt:ecj	11\\
%% org.apache.rocketmq:rocketmq-remoting	&io.netty:netty-all	9\\
%% org.eclipse.jdt:ecj	&net.sf.jasperreports:jasperreports	6\\
%% org.glassfish.jersey.core:jersey-common	&com.alibaba:fastjson	1\\
%% org.glassfish.jersey.core:jersey-common	&org.glassfish.hk2:hk2-utils	2\\
%% org.jetbrains.kotlin:kotlin-stdlib	&com.squareup.okio:okio	3\\
%% org.jsoup:jsoup	& us.codecraft:xsoup	2\\
%% org.mockito:mockito-core	&ognl:ognl	1\\
%% org.mockito:mockito-core	&org.springframework:spring-expression	1\\
%% org.mongodb:mongo-java-driver	&com.github.fakemongo:fongo	3\\



\paragraph{Service bypasses}
We find that all clients of libraries that use service loaders bypass the defined services. This is done most commonly through instantiation, but also through casts, subtyping and reflection.

For instance, component \emph{com.h2database:h2} advertises a service (\texttt{org.h2.Driver} implementing \texttt{java.sql.Driver}), making it a JDBC4-compliant driver. Its clients can obtain connections through the JDBC driver manager, which selects and instantiates instances based on driver URLs. However, client \texttt{glowroot.agent} (specifically class \texttt{org.glowroot.\-agent.embedded.util.DataSource}) directly instantiates \texttt{org.h2.jdbc.JdbcConnection} and therefore becomes dependent on using the particular database \emph{h2}. This leads to further direct calls to \texttt{org.h2.jdbc} APIs being observed. This is a clear bypass of a defined API. 

Now consider \emph{fastjson}. This component declares that it provides several services, including three services defined by interfaces in the \texttt{javax.ws.rs.ext} package, part of the JEE support for RESTful services. 
The respective services are implemented in classes in the package \texttt{com.alibaba.fastjson.support.jaxrs}.   
However, looking at clients, we also see calls to APIs provided in \emph{fastjson} APIs outside packages implementing the interfaces declared as services. For instance, \texttt{com.alibaba.fastjson.JSON::toJSONString} is called from \texttt{org.springframework:spring-web}.
%:4.3.7.RELEASE.    
This is a legitimate use of a JSON parser via a non-standard API, and in this sense, it is a false positive as we detect it as a API misuse. So \emph{fastjson} could be easily split into two components, one providing an ``open API'' for the JSON parser functionality, and one implementing and providing the RESTful services. The services would then depend on the open API. This would significantly reduce the overall API surface of fastjson, splitting it into two components, each with well defined functionalities and internal coherence.

We have mainly focussed on ``why not'', i.e. what mechanisms clients use to bypass API restrictions: reflection, violating conventions, and service loader bypasses, and reported results with respect to ``what thing''. When relevant, we discussed ``how'' the bypass occurred, especially for reflection, and we reported the directionality of the bypasses.

\begin{mdframed}[
  leftmargin=\parindent,
  rightmargin=\parindent,
  skipabove=\topsep,
  skipbelow=\topsep
  ]
{\bf Finding 1:} Programs are generally well-behaved: reflection is mostly used for serialization not API bypasses; OSGi and Java 9 module definitions are always respected; internal packages are only called sparingly; but service loaders are almost always bypassed.
\end{mdframed}

\subsection{Extent of API usage}
Continuing with our investigation of API surfaces, but moving from
mis-uses to uses, we present results on API use.
%, again using results obtained from our instrumentation.

Table~\ref{tab:usage-distribution} presents the usage distribution of
libraries' API elements (methods, fields, and classes subtyped) as
covered by client tests. The ``total in lib'' refers to the number of
public and protected members, including static ones. To calculate these totals,
we use the latest stable version of each library and consider it to be representative of 
the versions used by clients.  The ``distinct used'' column counts an
element once regardless of how many clients use that element, while
``total used'' counts an element once per client using it. We identify method calls by the
declared type of the receiver object, for example, calls to {\tt o1.f()} and {\tt o2.f()} are the same
if {\tt o1} and {\tt o2} have the same declared type.
Our uses 
include both vanilla and reflective uses of API elements.

% latex table generated in R 4.1.1 by xtable 1.8-4 package
% Sun Oct 24 16:55:02 2021
\begin{table*}[ht]
\centering
\caption{\label{tab:usage-distribution}Usage Distribution of API Elements by Clients: Method Invocations, Field Accesses, and Classes Subtyped.\\ Clients did not use library annotations.} 
\begingroup\scriptsize
\begin{tabular}{l!{\color{verylightgray}\vrule}rrrr!{\color{verylightgray}\vrule}rrrr!{\color{verylightgray}\vrule}rrrr}
& \multicolumn{4}{c!{\color{verylightgray}\vrule}}{Methods} & \multicolumn{4}{c!{\color{verylightgray}\vrule}}{Fields} & \multicolumn{4}{c}{Subtyping}\\
 & clients  &  total & distinct  & total   & clients   &  total & distinct   & total  & clients  & total & distinct   & total  \\ 
Library  & using & in lib & used & used & using & in lib & used & used & using & in lib & used & used
  \\ \arrayrulecolor{verylightgray}\hline
fastjson & 6 & 1631 & 13 & 16 & 3 & 512 & 5 & 6 & 0 & 190 & 0 & 0 \\ 
  commons-collections4 & 7 & 3561 & 17 & 19 & 2 & 142 & 9 & 9 & 1 & 358 & 1 & 1 \\ 
  commons-io & 8 & 1257 & 18 & 24 & 2 & 116 & 3 & 3 & 0 & 163 & 0 & 0 \\ 
  joda-time & 5 & 3406 & 31 & 33 & 3 & 202 & 2 & 3 & 0 & 145 & 0 & 0 \\ 
  gson & 7 & 679 & 70 & 110 & 4 & 79 & 3 & 4 & 3 & 60 & 4 & 6 \\ 
  json & 5 & 258 & 25 & 34 & 2 & 18 & 1 & 2 & 0 & 21 & 0 & 0 \\ 
  jsoup & 6 & 1032 & 79 & 117 & 0 & 142 & 0 & 0 & 2 & 100 & 3 & 3 \\ 
  slf4j-api & 46 & 378 & 68 & 180 & 0 & 22 & 0 & 0 & 7 & 29 & 9 & 25 \\ 
  jackson-databind & 15 & 675 & 88 & 115 & 12 & 376 & 35 & 48 & 12 & 148 & 20 & 35 \\ 
  jackson-core & 9 & 1996 & 124 & 154 & 9 & 611 & 39 & 65 & 6 & 117 & 12 & 15 \\ 
  h2 & 3 & 6712 & 8 & 8 & 0 & 1699 & 0 & 0 & 0 & 662 & 0 & 0 \\ 
\end{tabular}
\endgroup
\end{table*}


We can observe that at most 22\% (and on average 9\%) of the methods in the API surface
are used by clients, with \emph{slf4j-api} having the highest use proportion.
% , and sometimes less than 1\%  of methods. 
Table~\ref{tab:usage-distribution} suggests that there is a small overlap between methods
used by clients.

We also looked at the \emph{json} library and found that its own tests reach 151
of the 258 API methods, compared to the 25 methods used by our selected
clients. One possible reason for the sparsity of API use is that, at least in \emph{json}, over half of the API methods are overloaded, i.e. share a name with
some other API method.


% reads or writes would be nice to know, but we can't get it in time.
Despite standard OO practices calling for fields to be encapsulated, more than half of the libraries have their fields accessed by clients. Only 1 of the \emph{joda-time} and 25 of the \emph{jackson-core} fields are static (so it's not just constants); the rest of the accesses are to instance fields.
However, the number of fields used is often in the single digits,
the \emph{jackson} libraries being an exception with about 11\% of declared fields used.
About half of the libraries have clients subclassing a
single-digit number of library types, again with the exception of
\emph{jackson} with dozens of subtypes in clients.

We observe only one use of library annotations across all our benchmarks. \emph{crushpaper} uses \texttt{com.fasterxml.jackson.databind.annotation.JacksonStdImpl}, which is used for indicating implementation classes and is typically used by serializers and deserializers in \emph{jackson}.

Figure~\ref{fig:jaccard} presents a boxplot of
pairwise Jaccard similarities between the portion of
the APIs used by different clients for each library. Two clients $u_1$ and $u_2$ 
of library $L$ which use exactly the same parts of $L$'s API would result in a similarity of 1;
more generally, it is
\[ \frac{|\mbox{used.API}_L(u_1) \cap \mbox{used.API}_L(u_2)|}{|\mbox{used.API}_L(u_1) \cup \mbox{used.API}_L(u_2)|}. \]

Although Table~\ref{tab:usage-distribution} showed that some of the members overlap, we can
see from this Figure that the overall level of overlap is generally less than 10\%. We observe
a half-dozen client pairs (data points) where there is about 50\% overlap between client API uses, including for \emph{fastjson},
\emph{commons-io}, and \emph{json}. Although the mean overlap for \emph{slf4j-api} is near 0, there are also 4 pairs of clients which have more than 50\% overlap.

\begin{center}
\begin{figure}
\centering
\includegraphics[width=8cm, height=7cm]{./images/jac-sim-box-plot-declared}
\caption{\label{fig:jaccard}Jaccard similarities between clients' API usages}
\end{figure}
\end{center}
%% Digging into \emph{fastjson}, we found that it is a ``mixed type'' library: some clients use APIs related
%% to \texttt{javax.ws.rs}, the JEE standard for restful services, while others use just a standalone parser API that does not implement an external specification. 


We were not surprised by the generally small amounts of overlap, because the libraries' exposed API surfaces tended
to have thousands of exposed elements and hundreds of used elements. This result makes a convincing argument for our discussion about \emph{fastjson} and \emph{jsoup} in Sections~\ref{sec:fastjson-fission} and \ref{sec:jsoup-fission} how they could be fissioned into modules. Generally, across our benchmarks, small number of members are used by
multiple clients, but many others are used only once.

Table~\ref{tab:same-method} presents another view of API overlap between clients. We constructed the largest set
of methods shared by more than 1 client of a library (``max-set'') and report the size of that set as well as the percentage
of clients which use all of that set of methods. So, for \emph{json}, we can see that there is one method called
by 3 out of 4 clients, and for \emph{jsoup}, ten methods are all called
by 3 out of 4 clients. We characterize the amount of overlap as generally low but not
nonexistent: a few methods are repeatedly used.

% latex table generated in R 4.2.0 by xtable 1.8-4 package
% Mon Aug 29 11:02:02 2022
\begin{table}[ht]
\centering
\begingroup\small
\begin{tabular}{l!{\color{verylightgray}\vrule}rrr}
Library & Total No.  & \% Clients calling  & No. of methods  \\ 
 & of clients & same methods & called by (\%) clients \\
   \arrayrulecolor{verylightgray}\hline
fastjson & 26 & 50 & 1 \\ 
  commons-collections4 & 6 & 50 & 1 \\ 
  commons-io & 9 & 44 & 1 \\ 
  joda-time & 9 & 56 & 1 \\ 
  h2 & 3 & 67 & 13 \\ 
  gson & 17 & 41 & 1 \\ 
  json & 4 & 75 & 1 \\ 
  jsoup & 4 & 75 & 10 \\ 
  slf4j-api & 89 & 51 & 2 \\ 
  jackson-databind & 23 & 26 & 1 \\ 
  jackson-core & 13 & 46 & 4 \\ 
\end{tabular}
\endgroup
\caption{\label{tab:same-method}\% of clients calling same methods} 
\end{table}


%{\bf Research Question 2.} (a) Do clients often use only a subset of the published API? How big is this subset? (b) For each library, is that subset consistent across clients? (c) Do different libraries show different usage patterns by clients?

\begin{mdframed}[
  leftmargin=\parindent,
  rightmargin=\parindent,
  skipabove=\topsep,
  skipbelow=\topsep
  ]
{\bf Finding 2:} APIs are sparsely used by clients---mostly methods (9\% utilization), but a few fields (4\%) and supertypes (6\%). There is limited but nonzero overlap between the methods used by different clients.
\end{mdframed}
