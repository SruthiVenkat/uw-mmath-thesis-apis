% T I T L E   P A G E
% -------------------
% Last updated June 14, 2017, by Stephen Carr, IST-Client Services
% The title page is counted as page `i' but we need to suppress the
% page number. Also, we don't want any headers or footers.
\pagestyle{empty}
\pagenumbering{roman}

% The contents of the title page are specified in the "titlepage"
% environment.
\begin{titlepage}
        \begin{center}
        \vspace*{1.0cm}

        \Huge
        {\bf Studying and Leveraging API Usage Patterns}

        \vspace*{1.0cm}

        \normalsize
        by \\

        \vspace*{1.0cm}

        \Large
        Sruthi Venkatanarayanan \\

        \vspace*{3.0cm}

        \normalsize
        A thesis \\
        presented to the University of Waterloo \\ 
        in fulfillment of the \\
        thesis requirement for the degree of \\
        Master of Mathematics \\
        in \\
        Computer Science \\

        \vspace*{2.0cm}

        Waterloo, Ontario, Canada, 2022 \\

        \vspace*{1.0cm}

        \copyright\ Sruthi Venkatanarayanan 2022 \\
        \end{center}
\end{titlepage}

% The rest of the front pages should contain no headers and be numbered using Roman numerals starting with `ii'
\pagestyle{plain}
\setcounter{page}{2}

\cleardoublepage % Ends the current page and causes all figures and tables that have so far appeared in the input to be printed.
% In a two-sided printing style, it also makes the next page a right-hand (odd-numbered) page, producing a blank page if necessary.

% D E C L A R A T I O N   P A G E
% -------------------------------
\begin{center}\textbf{Author's Declaration}\end{center}
  % The following is a sample Delaration Page as provided by the GSO
  % December 13th, 2006.  It is designed for an electronic thesis.
  \noindent
This thesis consists of material all of which I authored or co-authored: see Statement of Contributions included in the thesis. This is a true copy of the thesis, including any required final revisions, as accepted by my examiners.

  \bigskip
  
  \noindent
I understand that my thesis may be made electronically available to the public.

\cleardoublepage

%  S T A T E M E N T  O F  C O N T R I B U T I O N S
% -------------------------------
\begin{center}\textbf{Statement of Contributions}\end{center}
   \noindent
This thesis consists mostly of chapters written for conference research paper submission
(1 rejected, 1 accepted), with some word changes, styling updates and other modifications.

Sruthi Venkatanarayanan was the sole author for Chapter~\ref{sec:applications}, which was written under the
supervision of Dr. Patrick Lam. Sruthi was responsible for developing the tool that performs the static and dynamic analyses, 
carrying out data collection and analysis and developing the VizAPI visualization tool.
Sruthi, Dr. Patrick Lam and Dr. Jens Dietrich were co-authors for Chapters~\ref{sec:introduction}, \ref{sec:related}, \ref{sec:apiusage} and \ref{sec:conclusion}.
  \bigskip
  
  \noindent
I understand that my thesis may be made electronically available to the public.

\cleardoublepage


% A B S T R A C T
% ---------------

\begin{center}\textbf{Abstract}\end{center}

Software projects make use of libraries extensively. Libraries have intended API surfaces—sets of exposed library interfaces that library developers expect clients to use. However, in practice, clients only use small fractions of intended API surfaces of libraries. Clients also use libraries in unexpected ways sometimes. Understanding usage patterns of library APIs by clients is beneficial to both client and library developers—targetting issues such as version upgrades, breaking changes and software bloating. We have implemented a tool to study both static and dynamic interactions between clients, the libraries they use, and those libraries’ direct dependencies. We use this tool to carry out a detailed study of API usage patterns on 90 clients and 11 libraries. We present a classification framework for developers to classify API uses. We then describe two additional developer-focussed applications of the data that our tool produces: a secondary visualization tool VizAPI, as well as the concept of library fission. Conceivably, VizAPI can allow client developers to answer the following query about the interaction of their code and the libraries they depend on: Will my code be affected by breaking changes in library APIs? Additionally, library developers can potentially find out about downstream code: Which APIs in my source code are commonly used by clients? The concept of library fission, by which we mean the splitting of libraries into sub-modules, is based on the usage patterns that we observe. This can potentially help library developers release backward compatible versions of their libraries. It could also help client developers isolate breaking changes and reduce the likelihood of vulnerabilities and version conflicts that may be introduced through direct or transitive dependencies. 

\cleardoublepage

% A C K N O W L E D G E M E N T S
% -------------------------------

\begin{center}\textbf{Acknowledgements}\end{center}
Firstly, I would like to thank Dr. Patrick Lam for patiently and constantly guiding me for the past two years. I have learnt perseverance and optimism from you, and this thesis would not have been possible without your ideas and encouragement.

I would also like to thank Dr. Jens Dietrich, our collaborator from the Victoria University of Wellington, for his invaluable inputs and for his multiple ideas to take this work in new interesting directions.

\cleardoublepage

% D E D I C A T I O N
% -------------------

\begin{center}\textbf{Dedication}\end{center}

This thesis is dedicated to my family.
\cleardoublepage

% T A B L E   O F   C O N T E N T S
% ---------------------------------
\renewcommand\contentsname{Table of Contents}
\tableofcontents
\cleardoublepage
\phantomsection    % allows hyperref to link to the correct page

% L I S T   O F   T A B L E S
% ---------------------------
\addcontentsline{toc}{chapter}{List of Tables}
\listoftables
\cleardoublepage
\phantomsection		% allows hyperref to link to the correct page

% L I S T   O F   F I G U R E S
% -----------------------------
\addcontentsline{toc}{chapter}{List of Figures}
\listoffigures
\cleardoublepage
\phantomsection		% allows hyperref to link to the correct page

% L I S T   O F   L I S T I N G S
% -----------------------------
\addcontentsline{toc}{chapter}{List of Listings}
\lstlistoflistings
\cleardoublepage
\phantomsection		% allows hyperref to link to the correct page

\cleardoublepage
\phantomsection		% allows hyperref to link to the correct page

% Change page numbering back to Arabic numerals
\pagenumbering{arabic}

