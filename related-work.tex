%======================================================================
\chapter{Related Work}
\label{sec:related}
%======================================================================

This work revolves around the usage of library APIs by clients. There is a large body of work investigating API usages. Mining API specifications was first introduced by Ammons et al~\cite{AmmonsETAL02MiningSpecifications}. In their work, frequent interaction patterns are obtained from program execution. More closely related to our present work is that by Zhong and Mei~\cite{zhong19:_empir_study_api_usages}, who have investigated API usages in a dataset of 7 experimental subjects and the libraries that they depend on. Some of those findings are relevant to us: they find that clients use less than 10\% of the declared APIs in libraries. Our results corroborate these findings. Our work differs from ~\cite{zhong19:_empir_study_api_usages} in that we are specifically not investigating sequencing relationships between API calls. Saied et al~\cite{saied15:_minin_multi_api_usage_patter} studied which API calls tended to co-occur in client code and inferred co-usage relationships between these calls. Our work goes beyond previous work by investigating not only which \emph{intended} APIs get used but also APIs which are not declared to be part of the interface but are used in practice. We record different types of usage patterns, which we explain further in Chapter \ref{sec:apiusage}.

Thummalapenta and Xie~\cite{thummalapenta08:_spotw} presented the related SpotWeb tool, which identifies framework hotspots (APIs that are often used) and coldspots (APIs that are never used); they do not consider framework APIs that are used but not intended to be. Hotspots and coldspots are, however, related to our investigations about client use of APIs; part of our study could be seen as investigating the prevalence of hotspots on our benchmark suite. Their notion of API usage is similar to ours, but they perform a static search to identify uses, while we statically record uses but also dynamically record test executions. They also identify the top $N$ percent of used APIs as hotspots, and unused APIs as coldspots. Viljamaa~\cite{viljamaa03:_rever_engin_framew_reuse_inter} also aimed to find hotspots but used concept analysis to do so.

Good API design is important, and Piccioni et al~\cite{piccioni13:_empir_study_api_usabil} carried out a detailed study to determine what contributed to API usability. Our work also takes an empirical approach and studies what clients use in practice, as well as parts of the API which are hidden from users and yet are still used. 

On the subject of API evolution, Yasmin et al~\cite{yasmin20:_first_look_deprec_restf_apis} investigate deprecation and removal of APIs in the RESTful context, while Zhou and Walker~\cite{zhou16:_api_deprec} investigate it in the context of Java APIs with examples on the Web, and Li et al~\cite{li18:_charac_deprec_android_apis} investigate deprecation in the Android context. Finding deprecated APIs is useful for preventing breaking changes. Calls to removed APIs will definitely fail, especially in the RESTful context, where the server retracts the API (there is no previous library version to fall back on). Internal APIs share some similarities with deprecated APIs, in that calls to deprecated APIs may fail in the future---there are no guarantees. Our tool records all client-to-library interactions, as well as intra-client and intra-library interactions, and so we record usage of internal APIs.

Our study of API usage patterns includes investigating API bypasses. Dynamic language features are one way to bypass protections,
and Dietrich et al~\cite{dietrich2017xcorpus} explored the extent of their use in the significant XCorpus. A related bypass technique
is unsafe Java (especially used for performance), and Mastrangelo et al~\cite{mastrangelo15:_use_your_own_risk} characterized the
use of unsafe Java in their benchmark suite.

Our goal is to investigate client uses of library code, targetting two applications: the VizAPI visualization tool and library fission. Our results can help both client and library developers. Clients benefit from sharpened warnings about unsafe upgrades, knowledge that some upgrades are safe, and having reduced attack surfaces. Library upgrades have been investigated by many researchers, including Lam et al~\cite{lam20:_puttin_seman_seman_version} and Kura et al~\cite{kula18:_do_devel_updat_their_librar_depen}. Kura et al found that most software had outdated dependencies, and that software developers had negative feelings about being required to constantly upgrade their libraries. Foo et al~\cite{foo18:_effic_static_check_librar_updat} proposed a static analysis which detected safe upgrades, but could only certify safety for 10\% of upgrades. Our combined static and dynamic approach presents the developer with more information. The applications of this information---the VizAPI visualization tool and library fission enables easier upgrades. From the library side, library fission can help with maintainability---smaller libraries can be released with fewer worries about downstream effects.

The related notion of ``tree-shaking'' has been well-known since at least the 1990s, and recently rebranded as ``debloating''.
The Jax project by Tip et al~\cite{tip02:_pract_extrac_techn_java} was early work on debloating in the Java space.
More recently, JShrink by Bruce et al~\cite{bruce20:_jshrin} has applied more modern techniques to programs using more modern Java features (for example, lambdas).
Both our approach and JShrink's integrate static and dynamic data to present recommendations to developers about what they should update and what they should exclude.
However, debloating depends on executing and analyzing the client. Library fission, on the other hand, does not require running analyses on clients for every execution, since it splits libraries based on client usage.

A related concept is library shading, as done for instance by the Maven shade plugin\footnote{\url{https://maven.apache.org/plugins/maven-shade-plugin/}}. We are looking for code that is included (due to indiscriminate inclusion of dependencies) but not used. A well-known problem that arises with dependencies is that different versions of a library can sometimes be included in the same client, and this is more likely to happen when there are more dependencies. Shading mitigates this problem, to some extent, by renaming included libraries such that each included version appears to be different.

Hejderup and Gousios~\cite{DBLP:journals/jss/HejderupG22} explore a question which is central to our approach---how well do client tests exercise their dependencies' libraries? To some extent, we rely on client test suites exercising enough of the dependencies to get valid results from our analyses. Their conclusion is that a combination of static and dynamic analysis of the client has some chance of detecting breaking changes in its dependencies, and we accordingly use static analysis to supplement our dynamic results.


% https://softwareengineering.stackexchange.com/questions/297276/what-is-a-shaded-java-dependency

Shah et al~\cite{shah13:_autom_depen_break_refac_java} present refactorings which enable dependency breaking. In our work, we instead use dynamic and static data to investigate library fission and enable clients to depend on the same libraries as before, but less of them, using techniques that are similar to library shading. The difference is that we create versions of libraries that are smaller, not simply renamed.

We now discuss existing work related to VizAPI, our visualization tool.
A representative tool from the software visualization literature is
CodeSurveyor, by Hawes et al~\cite{hawes15:_codes}, which visualizes large
codebases using the analogy of cartographic maps. While it
incorporates dependency information into the layout of the map, VizAPI
differs from CodeSurveyor in that VizAPI focusses on usage relationships
between different modules (for example, API invocations) using dynamic analysis by executing test
cases and static analysis to identify relationships between clients and libraries, rather
than investigating a particular system, as CodeSurveyor does.  Earlier
work includes the software cartography project by Kuhn et
al~\cite{kuhn10:_softw} and software terrain maps by DeLine~\cite{deline05:_stayin}.


Call graph visualization is, of course, a well-known technique, as seen for example, in the Reacher tool~\cite{latoza11:_visual_call_graph}. 
VizAPI also presents static and dynamic call information. However, we designed
it to support decisions about library/client interactions: the granularity of nodes is packages (typically it is methods);
and the layout is influenced by frequency of interactions. Another key difference is that VizAPI also displays interactions other than method calls, which include field usage, annotation usage, subtyping, reflective calls and dynamic proxies.

Kula et al~\cite{kula14:_visual_evolut_system_their_librar_depen} also developed a tool
to visualize changes in dependencies over time---but not how a particular client depends on its libraries. 
Our VizAPI tool's dependency visualizations will help developers
prioritize required upgrades as low-effort or high-effort.
 

Bergel et al~\cite{bergel14:_domain_specif_languag_visual_softw_depen_graph} propose the {\sc Graph} DSL
for software visualizations. That language could generate static representations similar to VizAPI's; however,
VizAPI chooses a specific point in the design space, and we argue that this point is useful for helping developers understand
potential impacts of upgrades.

